\subsection{Table piece}
\begin{itemize}
	\item \code{type} : le type de la pièce. Les valeurs possibles sont : Bureau, Salle de Cours, Autre.
	\item \code{gps\_lat} : latitude de la pièce.
	\item \code{gps\_long} : longitude de la pièce.
\end{itemize}

\subsection{Table personne}
\begin{itemize}
	\item \code{idPers} : l'identifiant de la personne.
	\item \code{nom} : le nom de la personne.
	\item \code{grade} : le grade de la personne. Les valeurs possibles sont : Etudiant, MCF, PU, BIATOSS, PE
\end{itemize}

\subsection{Table tache}
\begin{itemize}
	\item \code{date} : la date de la tâche.
	\item \code{tache} : la description de la tâche planifée. Les valeurs possibles sont : Enseignement, Recherche, Réunion.
	\item \code{idPers} : l'identifiant de la personne.
\end{itemize}

\subsection{Table appartient}
\begin{itemize}
	\item \code{idP} : l'identifiant de la pièce.
	\item \code{idPers} : l'identifiant de la personne.
\end{itemize}

\subsection{Table passePar}
\begin{itemize}
	\item \code{idP} : l'identifiant de la pièce.
	\item \code{idPers} : l'identifiant de la personne.
	\item \code{date} : la date du passage.
\end{itemize}

\subsection{Table reservation}
\begin{itemize}
	\item \code{idP} : l'identifiant de la pièce.
	\item \code{idPers} : l'identifiant de la personne.
	\item \code{date} : la date de la réservation.
\end{itemize}

\vspace{30px}
Les types des attributs peuvent être retrouvés dans le fichier de création des tables : \code{src/SQL/creation.sql}.