\subsection{Choix des clés}
Pour les tables \textbf{\code{piece}} et \textbf{\code{personne}} on choisit comme clés primaires \code{idP} et \code{idPers} car ils identifient clairement une pièce ou une personne. On choisit ces attributs comme clés primaires et ces attributs seront utilisés comme clés étrangères dans d'autres tables.\\

Pour les tables \textbf{\code{passePar}} et \textbf{\code{reservation}} nous devons définir comme clé primaire les attributs \code{idP}, \code{idPers} et \code{date} car on référence un événement qui se déroule à une date et qui lie une pièce à une personne. Il est nécessaire d'utiliser les 3 attributs comme clés primaires dans ces deux tables pour identifier un tuple. Cette tables sont des tables de liaison donc on utilise les clés primaires des tables que l'on veut lier.\\

Pour la table \textbf{\code{tache}}  nous devons définir comme clé primaire les attributs \code{tache}, \code{idPers} et \code{date} car on référence une tache planifiée par une personne à une date. L'explication est très similaire au paragraphe précédent. Cette tables sont des tables de liaison donc on utilise les clés primaires des tables que l'on veut lier.\\

Enfin, pour la table \textbf{\code{appartient}} on choisit comme clé primaire les attributs \code{idP} et \code{idPers} pour permettre de dire qu'une personne est propriétaire d'une pièce. Cette table est une table de liaison donc on utilise les clés primaires des tables que l'on veut lier.


\subsection{Choix des contraintes}
Parmi les contraintes qu'il nous a semblé primordial de respecter, la première était l'unicité des clés primaires qui devaient églament être non nulles. C'est pourquoi pour chaque table nous avons renseigné les clés primaires décrites précédemment par \code{PRIMARY KEY}.\\

Nous avons également veillé au respect des contraintes pour les clés étrangères en utilisant \code{references}. Ainsi, pour la table \code{tache}, nous avons indiqué que l'attribut \code{idPers} était une clé étrangère faisant référence à \code{idPers}, la clé de la table \code{personne}. De même, pour les tables \code{appartient}, \code{passePar} et \code{reservation}, \code{idP} et \code{idPers} sont deux clés étrangères correspondant respectivement aux clés primaires de \code{piece} et \code{personne}.\\
