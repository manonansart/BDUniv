\subsection{Choix des clés}
Pour les tables \textbf{\code{piece}} et \textbf{\code{personne}} on choisit comme clés primaires \code{idP} et \code{idPers} car ils identifient clairement une pièce ou une personne. On choisit ces attributs comme clés primaires et ces attributs seront utilisés comme clés étrangères dans d'autres tables.\\

Pour les tables \textbf{\code{passePar}} et \textbf{\code{reservation}} nous devons définir comme clé primaire les attributs \code{idP}, \code{idPers} et \code{date} car on référence un événement qui se déroule à une date et qui lie une pièce à une personne. Il est nécessaire d'utiliser les 3 attributs comme clés primaires dans ces deux tables pour identifier un tuple. Cette tables sont des tables de liaison donc on utilise les clés primaires des tables que l'on veut lier.\\

Pour la table \textbf{\code{tache}}  nous devons définir comme clé primaire les attributs \code{tache}, \code{idPers} et \code{date} car on référence une tache planifiée par une personne à une date. L'explication est très similaire au paragraphe précédent. Cette tables sont des tables de liaison donc on utilise les clés primaires des tables que l'on veut lier.\\

Enfin, pour la table \textbf{\code{appartient}} on choisit comme clé primaire les attributs \code{idP} et \code{idPers} pour permettre de dire qu'une personne est propriétaire d'une pièce. Cette table est une table de liaison donc on utilise les clés primaires des tables que l'on veut lier.


\subsection{Choix des contraintes}
\subsubsection{Clés primaires et étrangères}
	Parmi les contraintes qu'il nous a semblé primordial de respecter, la première était l'unicité des clés primaires qui devaient également être non nulles. C'est pourquoi pour chaque table nous avons renseigné les clés primaires décrites précédemment par \code{PRIMARY KEY}.\\

	Nous avons également veillé au respect des contraintes pour les clés étrangères en utilisant \code{references}. Ainsi, pour la table \code{tache}, nous avons indiqué que l'attribut \code{idPers} était une clé étrangère faisant référence à \code{idPers}, la clé de \code{personne}. De même, pour les tables \code{appartient}, \code{passePar} et \code{reservation}, \code{idP} et \code{idPers} sont deux clés étrangères correspondant respectivement aux clés primaires de \code{piece} et \code{personne}.\\

\subsubsection{Table des propriétaires}
	Concernant la table appartient, seules les personnes dont le grade est \code{MCF}, \code{PU} ou \code{BIATOSS} peuvent posséder une pièce et la pièce possédée doit être de type \code{Bureau}.\\

	Afin de vérifier que ces conditions sont bien respectées nous avons créé le trigger \code{triggerAppartient} qui appelle la fonction \code{testAppartient()}. Cette fonction vérifie que la personne n'est pas de type \code{Etudiant} ou \code{PE} et que la pièce est bien de type \code{Bureau}. Nous vérifions également dans cette fonction que la personne n'est pas déjà propriétaire de la pièce et qu'on ne tente pas de créer un troisième propriétaire. Cette dernière contrainte nous est imposée par le modèle Entité-Association.

\subsubsection{Table des réservations}
	La table réserve nécessite plusieurs contraintes : seules les personnes dont le grade est \code{MCF}, \code{PU} ou \code{BIATOSS} peuvent réserver une salle, il faut que la salle réservée ait un type cohérent avec le grade de la personne et la tâche, si celle-ci existe, pour laquelle elle est réservée. De plus la salle ne doit pas faire l'objet d'une autre réservation pour la même date. Nous vérifions chacune de ces contraintes à l'aide du trigger \code{triggerReservation}, correspondant à la fonction \code{testReservation()}.\\

	Dans un premier temps, cette fonction vérifie que la personne pour laquelle on veut faire la réservation est bien de grade \code{MCF}, \code{PU} ou \code{BIATOSS} à l'aide de plusieurs si. Puis la fonction vérifie qu'on ne tente pas d'effectuer une réservation dans le passé. Elle compte ensuite le nombre de réservation pour cette salle à cette date afin de vérifier que ce nombre est nul et qu'on peut donc bien réserver la salle.\\

	Dans une dernière partie, la fonction regarde si la personne a déjà planifié une tâche à cette date et si c'est la cas, que le type de la salle est bien cohérent avec celui de la tâche.

\subsubsection{Table des tâches}
	Enfin, un dernier trigger, \code{triggerTache}, associé à la fonction \code{testTache()}, vérifie les contraintes concernant la table tache, c'est-à-dire que la personne n'est pas de type \code{PE} et que la tâche planifiée est bien cohérente avec le grade de la personne.
