\section{Répartition du travail}
Retrouvez dans le tableau ci-dessous comment nous avons réparti entre nous les différentes fonctionnalités à réaliser pour mener ce projet : 
\begin{center}
	\begin{tabular}
	{| l ||	c |	c |} \hline
		Tâche & Manon Ansart & Antoine Augusti \\ \hline \hline
		\textbf{Conception de la base de données} & & \\ \hline
		Passage au modèle relationnel & \checkmark & \checkmark \\ \hline
		Création des tables & & \checkmark \\ \hline
		Insertion des données & \checkmark &  \\ \hline
		Création des droits & \checkmark &  \\ \hline
		Création des index & & \checkmark \\ \hline
		Création des fonctions & \checkmark &  \\ \hline
		Création des vues & & \checkmark \\ \hline
		Création des triggers & & \checkmark \\ \hline
		Script de suppression & \checkmark &  \\ \hline
		\hlineGras 
		\textbf{Application de gestion} & & \\ \hline
		Ajout d'une personne & \checkmark &  \\ \hline
		Suppression d'une personne & \checkmark &  \\ \hline
		Réservation d'une salle & & \checkmark \\ \hline
		Lieux visités par une personne & \checkmark &  \\ \hline
		Rapport d'activité d'une personne & & \checkmark \\ \hline
		Rapport d'intrusion & & \checkmark \\ \hline
		\hlineGras 
		\textbf{Rapport} & & \\ \hline
		Rédaction du rapport & \checkmark & \checkmark \\ \hline
		\hlineGras 
		\textbf{Gestionnaire de versions} & & \\ \hline
		Mise en place de Git &  & \checkmark \\ \hline
		\hlineGras 
	\end{tabular}
\end{center}
Bien évidemment ce tableau de répartition des tâches peut être vérifié à l'aide de nos commits sur notre répertoire de travail Git, se trouvant sur GitHub.

\section{Conclusion}
Nous avons été heureux de réaliser de réaliser ce mini-projet de bases de données tout d'abord car il a très bien repris toutes les notions que nous avons pu aborder cette année dans les cours de bases de données. Bien que la base de données à concevoir n'était pas très conséquente, nous avons pu revoir une grande partie des concepts que nous avons appris cette année.\\

De plus, nous avons été heureux de nous replonger dans nos précédents cours et travaux pratiques afin de savoir comment appliquer précisément la notion dont nous avions besoin au fur et à mesure de l'avancement de notre projet. Nous avons dû également nous référer plusieurs fois à la documentation en ligne de PostgreSQL pour accomplir des tâches que nous n'avions pas pu aborder en cours. Ce travail de recherche (qui mène à un succès !) a été stimulant.\\

Pour conclure, ce projet a été pour nous l'occasion d'appliquer toutes les notions apprises lors de séances de travaux pratiques. Nous sommes fiers et satisfaits au terme de notre projet de pouvoir interagir à l'aide de notre application avec la base de données que nous avons conçu. Toutefois nous n'oublions pas que ceci est un bien maigre résultat par rapport à ce que doit être une interaction efficace et agréable pour un utilisateur entre une base de données et des applications. Nous attendons avec impatience de pouvoir faire mieux !