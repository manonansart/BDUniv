\subsection{Rapport d'intrusion}
La fonctionnalité nous ayant posé le plus de difficultés a été le rapport d'intrusion. En effet, après avoir écrit une première version de la fonction \textit{estIntru} qui nous paraissait cohérente, nous avons constaté que les résultats obtenus ne correspondaient pas à ceux de l'exemple d'utilisation. \\

\inputminted[tabsize=4,linenos,fontsize=\small]{sql}{code/1.sql}

Les résultats que nous étions censés obtenir étaient :
\begin{center}
	\begin{tabular}
		{| l |	c |	c |} \hline
		2014-01-20 & s3 & Amanda Maire \\ \hline
		2014-01-20 & s3 & Mousse Line \\ \hline
		2014-01-20 & s2 & Mousse Line \\ \hline
		2014-01-13 & s1 & Louis Dort \\ \hline
		2014-05-10 & s7 & Geo Graff \\ \hline
		2014-01-22 & s1 & Albert Gamotte \\ \hline
		2014-01-20 & s3 & Amanda Maire \\ \hline
		2014-01-20 & s3 & Mousse Line \\ \hline
	\end{tabular}
\end{center}

Mais nous obtenions les résultats suivant :
\begin{center}
	\begin{tabular}
		{| l |	c |	c |} \hline
		2014-01-13 & s1 & Laure Aidubois \\ \hline
		2014-01-22 & s1 & Albert Gamotte \\ \hline
		2014-05-10 & s7 & Geo Graff \\ \hline
		2014-01-13 & s1 & Louis Dort \\ \hline
		2014-01-20 & s3 & Mousse Line \\ \hline
		2014-01-20 & s3 & Amanda Maire \\ \hline
	\end{tabular}
\end{center}



Le rapport d'intrusion résultant d'une fonction assez compliquée, il nous était très difficile de trouver la source de ces différences et d'expliquer nos erreurs en regardant simplement les tables. Nous avons donc été amenés à écrire une requête SQL nous permettant d'avoir une vue d'ensemble et de mieux voir quels tuples nous manquaient, quels tuples nous avions en trop et pourquoi.\\

Nous avons donc utilisé la requête :
\inputminted[tabsize=4,linenos,fontsize=\small]{sql}{code/2.sql}

qui nous a affiché les résultats suivants :

\begin{center}
	\begin{tabular}
		{| l |	c |	c | c | c | c |} \hline
		nom       & idpers &  grade   &    date    & idp & proprietaire \\ \hlineGras
		 Lance Pierre   & p1      & MCF      & 2014-01-12 & s1  & p1 \\ \hline
		 Lance Pierre   & p1     & MCF      & 2014-01-12 & s1  & p2 \\ \hline
		 Laure Aidubois & p2     & MCF      & 2014-01-13 & s1  & p1 \\ \hline 
		 Laure Aidubois & p2     & MCF      & 2014-01-13 & s1  & p2 \\ \hline
		 Louis Dort     & p7     & Etudiant & 2014-01-13 & s1  & p1 \\ \hline
		 Louis Dort     & p7     & Etudiant & 2014-01-13 & s1  & p2 \\ \hline
		 Bruno Zieuvair & p5     & PE       & 2014-01-14 & s1  & p1 \\ \hline
		 Bruno Zieuvair & p5     & PE       & 2014-01-14 & s1  & p2 \\ \hline
		 Mousse Line    & p9     & Etudiant & 2014-01-20 & s2  & p3 \\ \hline
		 Albert Gamotte & p3     & PU       & 2014-01-20 & s2  & p3 \\ \hline
		 Amanda Maire   & p10    & Etudiant & 2014-01-20 & s3  & p4 \\ \hline
		 Mousse Line    & p9     & Etudiant & 2014-01-20 & s3  & p4 \\ \hline
		 Albert Gamotte & p3     & PU       & 2014-01-22 & s1  & p1 \\ \hline
		 Albert Gamotte & p3     & PU       & 2014-01-22 & s1  & p2 \\ \hline
	\end{tabular}
\end{center}

Ainsi nous avons pu voir que \textit{2014-01-13 | s1 | Laure Aidubois}, considéré comme un intru par notre fonction, était la deuxième propriétaire de la salle s1 dans laquelle elle se trouve et ne devait donc pas être considéré comme intru. De ce fait \textit{2014-01-13 | s1 | Louis Dort}, présent au même moment, ne devait pas être considéré comme un intru non plus.

Nous en avons conclu que notre fonction ne gérait pas les propriétaires multiples pour un même bureau.\\

De même, nous avons constaté que le tuple \textit{2014-05-10 | s7 | Geo Graff}, présent dans notre rapport d'intrusion, n'apparaissait même pas dans les résultats que nous obtenions, et ceci pour une raison simple : la salle s7 étant une salle de cours, elle ne possède pas de propriétaire.

Nous avons donc pu voir que notre fonction ne gérait pas les pièces autres que les bureaux, qui ne possèdent pas de propriétaires et qui ne devaient donc pas être prises en compte dans le rapport d'intrusion.\\

À l'aide de cette requête nous avons donc pu repérer nos erreurs et effectuer les modifications nécessaires pour obtenir des résultats cohérents :\\

\inputminted[tabsize=4,linenos,fontsize=\small]{sql}{code/3.sql}