Nous avons dû installer sur nos ordinateurs personnels PostgreSQL en client et en serveur pour pouvoir développer rapidement. En effet, nous ne voulions pas à avoir à monter des tunnels SSH vers l'INSA de Rouen pour travailler sur le serveur PostgreSQL de l'INSA.\\

L'installation de PostgreSQL sous Ubuntu est facile mais la configuration est un petit peu plus délicate. Nous avons utilisé la documentation en ligne de PostgreSQL et quelques forums trouvés grâce à des moteurs de recherche. Nous avons mis un petit peu de temps à changer le mot de passe de l'utilisateur par défaut, créer notre utilisateur \code{grtt6}, créer une base du même nom et définir les droits adéquats dessus. Mais maintenant nous maîtrisons parfaitement la procédure ! Une fois ceci effectué nous n'avons plus eu aucun problème et nous avons pu écrire nos scripts SQL et les tester sans problème.