Ce mini-projet conclut les cours Bases de Données 1 et 2 de la première année du département ASI\footnote{ASI : Architecture des Systèmes d'Information.} de l'INSA de Rouen. Il vise à mettre en œuvre les concepts abordés lors de ces cours en réalisant une petite base de données et quelques fonctionnalités sur celle-ci. L'objectif final étant de proposer une application permettant d'interagir avec une base de données correctement modélisée.\\

\section{Processus de réalisation}
Le processus général pour mener la réalisation d'une base de données est ici simplifié et peut être décrit ainsi :
\begin{enumerate}[leftmargin=2cm]
	\item Passage d'un schéma Entité-Association à un modèle relationnel ;
	\item Définition des domaines, des conditions initiales et des droits ;
	\item Définition des index sur la base ;
	\item Création de fonctions et des vues sur la base ;
	\item Développement d'une application en C afin d'interagir avec la base de données conçue. 
\end{enumerate}

\section{Livrables attendus}
Les livrables attendus pour ce mini-projet sont les suivants :
\begin{itemize}
	\item Un rapport d'une dizaine de pages expliquant la démarche suivie et les choix de conception effectués ;
	\item 2 fichiers SQL : 
		\begin{itemize}
			\item \code{creation.sql} qui crée la base, les fonctions, les droits, les index et la remplit ;
			\item \code{suppression.sql} qui supprime tout ce qui a été créé.
		\end{itemize}
	\item Un fichier \code{appliUniv.pgc} qui contient l'application permettant d'interagir avec la base de données créée. Ce fichier est à compiler.
\end{itemize}